\chapter{手表上交互}

\quad\quad 在 Apple Watch 中,由于触摸屏的存在,大部分手表中的交互方式沿袭自带有触摸屏幕的智能手机。为了对手表中的交互方式进行备择设计,我们必须分析并且明确在 Apple Watch 中现存的交互方式及其优缺点。

\section{传统交互}

\subsection{点击}

\subsection{滑动}

\section{特有交互}

\subsection{Digital Crown}

Digital Crown 是苹果公司在Apple Watch上推出的一个全新的交互技术,苹果公司在退出此项交互的技术时,将其对比了在人机交互历史中的两个革命性的交互技术:鼠标和触摸屏,这意味着苹果公司认为,Digital Crown 是一项在手表上的革命性交互方式。

这种交互方式利用了传统手表的时钟旋钮在功能可见性上的不足。在传统手表中,时钟旋钮在普通状态下不具备任何功能,只有当旋钮被从里向外拉出时,才具备调节时间的功能,这一装置在大部分时间里都不能发挥自身的作用,是一个典型的需求驱动型设计,并没有仔细考虑过其自身的存在方式,只是习惯性沿用。

而在 Apple Watch 上,信息呈现的方式以流式进行总想展示,所有呈现的内容被限制在一个宽度固定、纵向可伸缩的屏幕区域里。这时,Digital Crown 便能发挥其旋钮的功能。当产生旋转时,内容在竖直方向上进行移动,从而呈现更多的内容;并且,在交互情景发生变化时,Digital Crown 能够表达出不同的交互指令,例如在影月播放界面时,Digital Crown 的旋转能够调节播放音乐的音量。

\subsection{Force Touch}

Force Touch 这项交互技术首次在民用消费品中出现,在学术界中,对触摸的感知被研究了多年,***等一系列文献研究了触觉感知如何在触摸屏上进行增强,包括感觉反馈、触摸面积的测量、触摸力度等等,而 Force Touch 就是触摸力度的实际体现。

Force Touch 一共将触摸行为分为了两个等级,第一触摸等级就是传统意义上的触摸行为,手指轻触屏幕时即可被感知;第二触摸等级就是 Force Touch,这时需要用户将触摸屏幕的力度提升到一个级别后,系统才会进行响应,进一步处理交互。

\subsection{Taptic Engine}

\section{其他交互}

\subsection{侧面按钮}

\subsection{常规手势}

\subsection{语音控制}
