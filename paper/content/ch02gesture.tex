\chapter{手表上交互}

\quad\quad 在 Apple Watch 中,由于触摸屏的存在,大部分手表中的交互方式沿袭自带有触摸屏幕的智能手机\cite{WatchGuidelines:2016}。为了对手表中的交互方式进行备择设计,我们必须分析并且明确在 Apple Watch 中现存的交互方式及其优缺点。

\section{传统交互}

传统交互上苹果公司认为在 Apple Watch 的小屏幕上实施超过两个接触点的点按行为是严重影响用户交互的,因此 Apple Watch 上的触摸屏没有沿用多点触控的方案。并且,在屏幕的普通操作中,只有基础点按和四个方向上的滑动。

\subsection{点按}

屏幕的上的普通单点触控与智能手机上的方式并没有明显差异。这种点按方式在人机交互形式上早已被大众所接受,此种方式的用户接受成本也是最低最自然的。

然而将此种交互应用在智能手表上时则极大的降低了其自身的可用性,根据触摸屏上的 FFitts' 定律\cite{Bi:2013:FLM:2470654.2466180}:

\begin{equation}
T=a+b\log_{2}{\left(\frac{A}{\sqrt{2\pi e(\sigma^2-\sigma_{a}^2)}}+1 \right)}
\end{equation}

其中 $\sigma$ 是触摸点分布的标准差,$\sigma_a$是输入手指的绝对精度。$A$为开始点到目标中心的距离。

根据 FFitts' 定律我们可以看到,一方面,在屏幕大小及其有限的屏幕下将另一只手的手指一动到手表屏幕上会使得$A$的值很大,而且正由于屏幕大小的限制,$\sigma$的值不会较大甚至比手机触摸屏上的标准差还小,而对于同一目标而言$\sigma_a$的值又不存在变化。因此 $T$ 值会明显变大,即手表屏幕上的点按交互并不利于交互。

\subsection{滑动}

屏幕上点按位置的连续变化形成了滑动式的交互。Apple Watch

\section{特有交互}

Apple Watch 在传统触摸屏交互的基础上引入了三个全新的交互硬件,分别是Digital Crown、Force Touch 和 Haptic Engine。

\subsection{Digital Crown}

Digital Crown 是苹果公司在Apple Watch上推出的一个全新的交互技术,苹果公司在退出此项交互的技术时,将其对比了在人机交互历史中的两个革命性的交互技术:鼠标和触摸屏,这意味着苹果公司认为,Digital Crown 是一项在手表上的革命性交互方式。

这种交互方式利用了传统手表的时钟旋钮在功能可见性上的不足。在传统手表中,时钟旋钮在普通状态下不具备任何功能,只有当旋钮被从里向外拉出时,才具备调节时间的功能,这一装置在大部分时间里都不能发挥自身的作用,是一个典型的需求驱动型设计,并没有仔细考虑过其自身的存在方式,只是习惯性沿用。

而在 Apple Watch 上,信息呈现的方式以流式进行总想展示,所有呈现的内容被限制在一个宽度固定、纵向可伸缩的屏幕区域里。这时,Digital Crown 便能发挥其旋钮的功能。当产生旋转时,内容在竖直方向上进行移动,从而呈现更多的内容;并且,在交互情景发生变化时,Digital Crown 能够表达出不同的交互指令,例如在影月播放界面时,Digital Crown 的旋转能够调节播放音乐的音量。

\subsection{Force Touch}

Force Touch 这项交互技术是首次在民用消费品中出现。在学术界中,对触摸的感知被研究了多年,***等一系列文献研究了触觉感知如何在触摸屏上进行增强,包括感觉反馈、触摸面积的测量、触摸力度等等,而 Force Touch 就是触摸力度的实际体现。

Force Touch 一共将触摸行为分为了两个等级,第一触摸等级就是传统意义上的触摸行为,手指轻触屏幕时即可被感知;第二触摸等级就是 Force Touch,这时需要用户将触摸屏幕的力度提升到一个级别后,系统才会进行响应,进一步处理交互。

\subsection{Haptic Engine}

\section{其他交互}

\subsection{侧面按钮}

\subsection{常规手势}

\subsection{语音控制}
