
\appendix

  \chapter{\heiti 使用许可}
  \label{appendix:a}

  此毕业设计的程序、论文等全部源文件均已开放源代码,可以在论文主页:\url{https://changkun.us/bechelorthesis/}或 GitHub 仓库中查看:\url{https://github.com/changkun/BachelorThesis/}

  注意,本毕业设计(论文)中的所有文字、图片及视频内容均采用知识共享署名-非商业性使用-相同方式共享(BY-NC-SA) 4.0 国际许可协议\footnote{\url{http://creativecommons.org/licenses/by-nc-sa/4.0/}}进行许可。

  其余部分(代码)则采用 GNU Public Licence v3\footnote{\url{http://www.gnu.org/licenses/gpl-3.0.en.html}} 进行许可。

  \cleardoublepage

  \chapter{\heiti 用户调研问卷 - 第一部分}
  \label{appendix:b}

  您好!

  感谢您参与我们的研究!这份问卷是我们本次研究的第一部分,我们需要您填写这份简单的问卷。
  \textbf{其中用*标注的问题是必须填写的}。

  \begin{enumerate}
      \kaishu
      \item \textbf{请选择下面你觉得操作最舒适的多个操作*}\\
      请选择至少五个选项
      \begin{itemize}
          \item 大拇指在食指上滑动
          \item 大拇指在中指上滑动
          \item 大拇指在无名指上滑动
          \item 大拇指在小指上滑动
          \item 大拇指和食指的快速捏合
          \item 大拇指和食指的长时间捏合
          \item 大拇指和中指的快速捏合
          \item 大拇指和中指的长时间捏合
          \item 大拇指和无名指的快速捏合
          \item 大拇指和无名指的长时间捏合
          \item 大拇指和小指的快速捏合
          \item 大拇指和小指的长时间捏合
          \item 五个手指从伸展到握拳状态
      \end{itemize}
      \item \textbf{当你在马路、公交车、地铁等公共场所看到陌生人抬起手进行问题 1 的操作时,你会怎么想?}\\
      请在下方写下你的想法
      \item \textbf{请连接下方左右侧的你认为的所有符合你操作直觉的两个选项*}\\
      左侧的每个选项只能连接一个右侧选项,若没有则可不进行连接

      \begin{tabular}{ l r }
       ~ & 大拇指在食指上滑动 \\[-3pt]
       ~ & 大拇指在中指上滑动 \\[-3pt]
       ~ & 大拇指在无名指上滑动 \\[-3pt]
       ~ & 大拇指在小指上滑动 \\[-3pt]
      手指执行轻点操作(如手机屏幕) & 大拇指和食指的快速捏合 \\[-3pt]
      手指用力按压某物 & 大拇指和食指的长时间捏合 \\[-3pt]
      调节旋钮 & 大拇指和中指的快速捏合 \\[-3pt]
      手指在平面上向左滑动 & 大拇指和中指的长时间捏合 \\[-3pt]
      手指在平面上向右滑动 & 大拇指和无名指的快速捏合 \\[-3pt]
       ~ & 大拇指和无名指的长时间捏合 \\[-3pt]
       ~ & 大拇指和小指的快速捏合 \\[-3pt]
       ~ & 大拇指和小指的长时间捏合 \\[-3pt]
       ~ & 五个手指从伸展到握拳状态 \\[-3pt]
      \end{tabular}
  \end{enumerate}

  % \centerline{

  % }

  \cleardoublepage


  \chapter{\heiti 用户调研问卷 - 第二部分}
  \label{appendix:c}
  您好!

  感谢您参与我们的研究!这份问卷是我们本次研究的最后一个部分,我们需要您填写这份简单的问卷。
  \textbf{其中用*标注的问题是必须填写的}。

  \begin{enumerate}
      \kaishu
      \item \textbf{您的参与编号*}\\
      请在下方的空白处填写
      \item \textbf{年龄?}\\
      请在下方的空白处填写
      \item \textbf{性别?*}\\
      请在下方选择合适的选项
      \begin{itemize}
          \item 男
          \item 女
      \end{itemize}
      \item \textbf{佩戴手表的频率?*}\\
      请对下方的问题以:从不、极少、有时、经常、总是,这五个选项来回答。
      \begin{itemize}
          \item 将手表佩戴在左手的频率?
          \item 将手表佩戴在右手的频率?
          \item 不佩戴手表的频率?
          \item 忘记带手表的频率?
      \end{itemize}
      \item \textbf{是否有智能手表的使用经验?*}\\
      请在下方选择合适的选项
      \begin{itemize}
          \item 是
          \item 否
      \end{itemize}
      \item \textbf{使用过什么型号的智能手表?*}\\
      如果5中勾选了『是』,请在下方的空白处填写
      \item \textbf{拥有多少款不同的手表?*}\\
      包括智能手表和传统手表,不同的表带按不同款计算, 即使不经常使用手表,请在下方的空白处填写数字
      \item \textbf{是否还在继续使用手表或智能手表?*}\\
      请在下方选择合适的选项
      \begin{itemize}
          \item 是
          \item 否
      \end{itemize}
      \item \textbf{原因是什么?*}\\
      请在下方的空白处填写
      \item \textbf{是否考虑购买智能手表?*}\\
      请选择下方合适的选项
      \begin{itemize}
          \item 是
          \item 否
      \end{itemize}
      \item \textbf{原因是什么?*}\\
      请在下方的空白处填写
      \item \textbf{你觉得最影响你使用智能手表的因素是什么?*}\\
      请选择下方合适的一个选项
      \begin{itemize}
          \item 价格
          \item 电池续航
          \item 使用习惯
          \item 其他,请写下你的原因
      \end{itemize}
      \item \textbf{对本次实验你是否还有其他疑问、看法,欢迎您提一些意见。}\\
  \end{enumerate}
