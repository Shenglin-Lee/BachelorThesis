
\begin{abstract}{智能手表;非接触式;手势交互;备择设计}

\phantomsection\addcontentsline{toc}{chapter}{摘要}     % 向目录中添加摘要索引

本文以现有智能手表产品的代表 Apple Watch 为例,对 Apple Watch 上的交互模式的优缺点进行了全面的分析,并据此给出了一套非接触式的备择设计。其配合了 Haptic Engine 对用户的直观震动反馈,完成了对基础点按、滑动、Digital Crown、Force Touch等系列原生交互的非接触式手势交互设计,此设计将嵌套两步逻辑的十种原生交互简化为了单步逻辑上的八种备择交互,且消除了接触式交互的依赖,解决了现有交互中对双手依赖的缺陷,并同时说明了给出的备择设计的交互完备性。通过用户调研,对本套交互方式的设计进行了评估,结果显示该方案的操作逻辑和舒适度良好。在最后,本文对设计的硬件结构、交互方式和系统架构的现有缺陷进行了讨论,并从中得到的启示,给出了可行的解决思路。
\end{abstract}
\cleardoublepage
\phantomsection\addcontentsline{toc}{chapter}{Abstract} % 向目录中添加 Abstract 索引
\begin{abstractEng}{Smart Watches; Contact-free; Gesture Interaction; Alternative Design}
In this thesis, we explore Apple Watch as the representative of smart watches, and analysed the advanteges and disadvanteges of interaction pattern in Apple Watch with watchOS 2. According to this, we introduce a contact-free alternative design for smart watch interaction.

This design, combine with the Haptic feedback, convert the basic tap, swipe, Digital Crown, Force Touch and etc. native watch interaction to a contact-free gesture interaction. It simplefied ten native interaction with two-step logic to only eight alternative interaction with sigle-step logic, eliminated contact-required interaction method, solved the bimanual interaction within smart watches, and explained the completeness of this alternative design.

Through user study, we evaluate these alternative designs and the result shows  the interaction operate logic and gesture confortable all acceptable.

Finally, we discussed the weakness of current hardware structure, interaction design and system architecture. Trough this, we proposed a few implications for the future improvements.
\end{abstractEng}
