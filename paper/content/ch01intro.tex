\chapter{概述}

穿戴式设备兴起于二十一世纪一零年代初期,伴随计算机芯片技术三十几年的快速发展,终于在民用消费品市场上出现了以手表、手环为代表的几个主要表现形式。对于依然存在的性能上的不足,这给穿戴式设备的软件设计带来了巨大的挑战和机遇\cite{DBLP:journals/corr/JiangCZZKZ15}。

\section{目的、背景及意义}

【本节接下部分的内容待写】

从机器被制造\cite{dongshihai2004}

而对于

人机交互的研究一直都走在技术实现的前面,在实现人机交互\cite{hudson2014concepts}需要客服很多

从交互的表现上看,可以将交互分为显式交互和隐式交互,文\cite{yuweining2004}。

\section{相关工作}

手表上的大部分交互都依赖两只手来完成,我们要对现有的设计进行重新设计,并解除对双手接触式交互的依赖,考虑到手表需要抬碗才能观看的自身属性,对于交互的设计便落在了佩戴手腕的那只手上,这也就无法绕开对手势技术的探讨。

【本节接下部分的内容待写】

手势技术在人机交互的研究中一直经久不衰,手势可以按空间形式分为平面手势和空间手势。对于平面手势而言,已经有较为成熟的 $\$1$ \cite{wobbrock2007gestures}和 $\$n$ \cite{anthony2010lightweight}算法;而对于空间手势而言,最大的困难就是如何确定手势的开始和结束,幸运的是,与手表的交互在抬起腕臂后,由手表输出给用户的全部主要信息都是通过表盘,这时被确定在佩戴手表的手臂的移动不能幅度过高,因此这种交互也和传统的空间手势不同
