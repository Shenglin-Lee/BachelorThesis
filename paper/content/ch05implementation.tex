\chapter{编码实现}

\section{环境搭建}

无论是手表端的还是服务端,都存在框架依赖,因此环境搭建不可避免。在本项目中,服务端使用 NodeJS 进行编码,限于篇幅,对 NodeJS 相关的基本环境,如 Node 本体,NPM 包管理等常见工具的配置在文中略去,这里主要介绍运行本平台最重要第三方 Leap Motion 环境;而在手表端中,虽然我们不依赖其他第三方框架,但由于 watchOS 自身的限制\cite{WatchConnectivity:2016} 在 watchOS 2 中我们需要使用 WatchConnectivity 框架与 iOS 应用本体进行数据通信。
%而又由于 iOS 系统本身限制(iOS 9及以上)强制要求应用必须与 HTTPS 服务器进行通信,因此这里介绍在 iOS 9 中与 HTTP 服务器通信的配置方法。

\subsection{配置本地 LeapMotion 环境}

LeapMotion 提供了在 Mac OS X 中的开发环境,并且提供了各种不同的开发语言,根据上文的讨论,我们需要在服务端配置 LeapMotion 本体以及 LeapJS。

\textbf{1. 安装}

首先,在应用程序中引入 Leap SDK 需要在相应桌面端安装 LeapMotion 宿主程序,进而才能使用 LeapMotion 相关的 API。

其次,在 Node App 的 package.json 中添加 LeapJS 依赖:
\begin{lstlisting}
{
    "dependencies": {
        "leapjs": "^0.6.4"
    },
}
\end{lstlisting}
再使用 npm install 安装 LeapJS。

\textbf{2. 环境配置}

受到 LeapMotion 自身的限制\cite{Leap:2016},WebSocket 服务并非默认的向非本地访问开放,因此需要将 Leap 配置启用非本地客户端连接。

这需要对 LeapMotion 的配置文件进行修改。在修改配置之前,需要关闭 LeapMotion 的相关服务。在 Mac 中,使用下面的命令关闭 LeapMotion 的守护进程:
\begin{lstlisting}
sudo launchctl unload /Library/LaunchDaemons/com.leapmotion.leapd.plist
\end{lstlisting}

接下来我们需要修改 LeapMotion 的配置文件,根据 LeapMotion 的官方文档显示,Leap 包含两个不同的配置,其中控制面板配置的优先级最高,因此我们需要下面这个目录下:
\begin{lstlisting}
$HOME/Library/Application\ Support/Leap\ Motion
\end{lstlisting}

找到 config.json 的修改配置,编辑 config.json 文件,并在 configuration 字段中的任意位置添加一条:
\begin{lstlisting}
"websockets_allow_remote": true
\end{lstlisting}

最终得到:

\begin{lstlisting}
{
    "configuration": {
        "websockets_allow_remote": true,
        "background_app_mode": 2,
        "images_mode": 2,
        "interaction_box_auto": true,
        "power_saving_adapter": true,
        "robust_mode_enabled": false,
        "tracking_tool_enabled": true
    }
}
\end{lstlisting}

保存退出,重新启动 LeapMotion 服务,便完成了 LeapMotion 的相关配置:

\begin{lstlisting}
sudo launchctl load /Library/LaunchDaemons/com.leapmotion.leapd.plist
\end{lstlisting}

\subsection{配置 watchOS 的网络访问}

\section{服务端编码}

实现一个端口为 10086 的 HTTP 服务器是第一步:
\begin{lstlisting}
var http = require('http');
function handler (req, res) {
    res.writeHead(200);
}
http.createServer(handler).listen(10086);
console.log("Server running at http://locoalhost:10086")
\end{lstlisting}

下面我们来关注服务端中对 LeapMotion 手势的关键处理。

\subsection{Tap 手势识别}

\subsection{Swipe 手势识别}

\section{手表端编码}
