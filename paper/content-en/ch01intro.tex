\chapter{Introduction}

With 30-year rapid development of computer chip technology, wearable devices sprang up in the 2010s, mainly represented by smart watches and smart bands in civil electronics market.
With regard to smart bands, they are only collecting data of users when wearing them. In essence, there does not exist direct interaction between them and users.
As for smart watches, despite that they are installed with a large number of operating systems with simplified and appropriate designs, there still exist performance weaknesses, thus bringing great challenges and opportunities for the software design of wearable devices\cite{DBLP:journals/corr/JiangCZZKZ15}.

%可穿戴设备真正兴起于二十一世纪一零年代初期,伴随计算机芯片技术三十几年的快速发展,
%终于在民用电子消费品市场上出现了以手表、手环为代表的几个主要表现形式。
%对于手环而言,仅仅只采集用户佩戴时的使用数据,本质上而言几乎和用户不存在直接交互。
%而对于智能手表来说,尽管搭载了大量简化、合适设计的操作系统,但依然存在的性能上的不足,
%这便给穿戴式设备的软件设计带来了巨大的挑战和机遇\cite{DBLP:journals/corr/JiangCZZKZ15}。

\section{HCI \& Wearable Computing}

Since the middle and late 1990s, the concept of wearable devices has gradually come into people’s view.
%从上个世纪九十年代中后期起,可穿戴设备的概念开始逐渐进入人们的视野。

Wearable devices range from simple input devices(such as smart bands) to complicated devices with operating systems and communication functions.
Wearable devices are designed to provide computing devices more intimate than smart phones for human beings. In essence, wearable computing is based on this kind of small and wearable computers. “Always-on” is a distinctive feature of wearable computing. Wearable devices are worn on a certain part of users to perceive their environment and observe their behavior.
%一个可穿戴设备可以从简单的输入设备(如手环)到复杂的具备操作系统和通信功能的设备(如 Apple Watch、Android Wear 设备)。可穿戴设备出现的初衷就是为了向人们提供比智能手机更亲密的、一个计算设备。而穿戴式计算本质上就依托于这样一种小型的、可穿戴的计算机。『实时在线』是穿戴式计算的显著特点,可穿戴设备穿戴于用户的某个部位,时刻感知着用户环境以及观察用户的行为。

Wearable computing aims at providing people with personal assistant services, such as convenient access to the Internet, event notification and information collection. This ubiquitous computing technology based on wearable computing integrates multi-modal interaction, context-interactive model and augmented reality into one, so it is the ultimate task desired in the field of human-computer interaction\cite{dongshihai2004,yuweining2004}.
%穿戴式计算的目标是提供一个类似个人助理的服务,例如便捷的互联网访问、事件通知、信息采集等等,这种基于穿戴式计算的普适计算技术集多通道交互(Multi-Modal Interaction)、上下文感知交互模型(Context-Interactive Model)和增强现实(Augmented Reality)等于一身,是人机交互领域所期望的终极任务。

However, limited to current level of science and marketing, these researches are still on a primary stage. Above all, when reflecting on the reasons for adding a device on themselves, people would find there it makes little sense to wear them, so wearable devices could be easily substituted. Even so, researches on human-computer interaction of wearable devices are still on the way and have gone ahead of electronics. Although there is still a long way to go to realize these designs and that they might never be put into used, these research methods and concepts arising from them are of significance for people to continue pondering\cite{hudson2014concepts} and reflecting and promote the progress of the industry step by step.
%然而受到当下科学水平、市场推广的限制,这些研究依然还非常基础,尤其当人们回过头来思考为什么要在自己身上增加一个设备时,可穿戴设备的存在意义几乎为零,可替代性非常高。即便如此,对于可穿戴设备的人机交互相关研究一直在继续,并且走在电子消费品的前面,尽管离实现这些设计还很遥远,甚至可能永远不会被应用,但这些研究的方法以及它们产生的概念,都有着其存在的让人们继续思考、反思的价值\cite{hudson2014concepts},并一步步推动着这个行业的进步。

\section{Subject Significance}

Nowadays, the development of smart phones has been in a downturn, but wearable devices fail to gain great momentum as desired. Regardless of their acceptability, usability and pricing, smart watches do not have obvious significance of existence, for their usages are much related to the contexts they are in.
%在智能手机的发展已经略显疲态的今天,而可穿戴设备却显得力不从心。无论从其产品的可接受度、易用性还是价格,智能手表存在的意义还不够显著,智能手表的使用与它本身所处的情景很有关系\cite{zhuzijian2015}。

Therefore, the following issues must be considering carefully before design alternative interaction:
%因此,在对智能手表的交互进行设计前,要考虑以下几点:

\begin{enumerate}
    \kaishu
    \item How to insure interaction correspond human intuition and comfort level?
    %如何将交互方式做到足够符合人类的操作直觉与舒适;
    \item How to erase sense of a separate existence(Social Acceptability)?
    %如何消除交互形式与社会人文的隔阂感(社会接受度);
    \item How to balance affordance and system complexity?
    %如何平衡一个交互设计的功能可见性和系统复杂性。
\end{enumerate}

This thesis explore Apple Watch as the representative of smart watches, considering watchOS features, designed a alternative interaction modality which can be contact-free for watch interaction. Within this interaction pattern, user target can be more significant, smart watch application scenarios gains its extend and the complexity of interaction logic can be simplified so that the meaning of smart watches will be indirectly increase.
%本文考虑了目前智能手表市场为代表的产品——Apple Watch,结合其操作系统自身特点,设计另一种无接触的、可以释放双手依赖的交互模式。在这种交互模式下,用户在手表上执行交互的目标得到放大。手表的应用情景能够被得到放大,交互逻辑的复杂度得到简化,间接增加了智能手表的存在意义。

\section{Related Works}

Smart watch interface is the most intuition part of the interaction\cite{liuheng2015}, research on interfaces has been more mature and in-depth\cite{chengshiwei2009,fuaiming2006}. Apple Watch as a representative is the most simple, excellent interfaces, and its Human Interface Guidelines\cite{WatchGuidelines:2016} has become the standard guidelines of smart watches.
%智能手表中的界面是与人进行交互中最直观的部分\cite{liuheng2015},关于界面的研究已经较为的成熟和深入\cite{chengshiwei2009,fuaiming2006},其中为代表的 Apple Watch 是在所有智能手表中对界面最简洁、优秀的,其发展出来的『人机交互设计则例』\cite{WatchGuidelines:2016}已经成为交互设计中的标准指南。

However, even in Apple Watch, user still need two hands to performing their actions. So, if we have to redesign the whole interaction and release the bimanual interaction required on smart watches, then we have to discuss gesture techniques due to we only have one hand to perform interactions at this time.
%然而在执行交互输入的部分却依然混乱不堪,市面上所有的智能手表上的交互都依赖两只手来完成,考虑到手表需要抬碗才能观看的自身属性,
%如果我们要对现有的设计进行重新设计,并解除对双手接触式交互的依赖,
%对于交互模式的设计便落在了佩戴手腕的那只手上,这也就无法绕开对手势技术的探讨。

\subsection{Gesture Interaction}

Gesture as a technique in human-computer interaction has been enduring, it can be cetegorise as 2D gestures and 3D gestures.
For 2D gestures there are $\$1$ \cite{wobbrock2007gestures} and $\$n$ \cite{anthony2010lightweight} algorithms are maturing and the main idea is resampling gesture point in same interval, then use 2-norm to caculate and compare gesture series.
For 3D gestures, the hardest part is to confirm where gesture starts and ends, however most gesture recognize algorithm always avoid this problem by using recognize a specific gesture or construct a state machine of gesture\cite{liuqingshui2002,chenyaxi2014,dihaijin2011,houwenjun2015,Vatavu:2014:LGT:2602299.2602316}. Fortunately, interaction on smart watches always after user rise their hand, all information output through display ensures a fact that the amplitude of arm could not too much or it will abstract users. Thus, interactions on smart watches could be gesture manipulated but not identical.
%手势技术在人机交互的研究中一直经久不衰,手势可以按空间形式分为平面手势和空间手势。
%对于平面手势而言,已经有较为成熟的 $\$1$ \cite{wobbrock2007gestures}
%和 $\$n$ \cite{anthony2010lightweight}算法,这些方法将连续的手势在等时间间隔内进行重新采样,然后利用二范数对两个不同的手势序列进行对比;
%而对于空间手势而言,最大的困难就是如何确定手势的开始和结束,但是实际上由于手势的特殊性我们避开这些问题,通过识别特定手势、对手势状态机进行建模依然能够进行相关的应用\cite{liuqingshui2002,chenyaxi2014,dihaijin2011,houwenjun2015,Vatavu:2014:LGT:2602299.2602316}。幸运的是,与手表的交互在抬起腕臂后,
%由手表输出给用户的全部主要信息都是通过表盘,这时被确定在佩戴手表的手臂的移动不能幅度过高,
%因此这种交互也和传统的空间手势并不完全相同。

L. Christian et al. \cite{loclair2010pinchwatch} propused a complex gesture set of pinch for user to operate smart watches. Although it's purpose as same as ours, but their pinch gesture required redesign whole interaction logic on smart watches, basiclly, the circumstances is unrealistic.
%文\cite{loclair2010pinchwatch}详细研究并定义一套复杂地、不同地捏合手势来完成不同的单手、无需用户进行视觉观察的交互任务。虽然该文实现的效果与本文的目的有一部分相同,但可惜的是,这些复杂的捏合手势需要将整个手表的交互逻辑重新设计,这于情于理都不够现实。

For research on contact-free interaction, \cite{lv2015extending} use a RGB camera attached on Google glass and applied vision method to recognize gesture, but this method is only appropriate for classes user not everyone.
%非接触式的交互设计同样有相关的研究,文\cite{lv2015extending}使用 Google 眼镜上的摄像头应用视觉方法来检测用户的手势,进而完成非触摸式的交互,然而文中只是对这一技术只适合有佩戴眼镜需求的用户,不具备普遍适用的价值。
Kerber et al. \cite{Kerber:2015:WPM:2836041.2836063} even expect to use diverse ortation of arm.
%而文\cite{Kerber:2015:WPM:2836041.2836063}甚至考虑了利用所佩戴手表的手臂的旋转方式来制造不同的交互。

In these research, the interaction gains its augmentability, nonetheless they have ever prepense these techniques applicable existence form.
%在这些研究中,现有的手表交互虽然得到了扩展,将手势相关的成熟研究应用到智能手表上,
%但这些研究并没有认识到手表上的交互方式就存在不够便捷的根本问题。

\subsection{Extension Interaction}

Some research inspire us gives various idea, they expand the interaction outthrough watch itself, they put interaction around the watch surface\cite{Knibbe:2014:EIS:2559206.2581315,Kratz:2009:HEA:1613858.1613912}, band\cite{Perrault:2013:WSG:2470654.2466192}, even skin\cite{Ogata:2015:SSG:2735711.2735830}. \cite{kim2007gesture} puts sensors around the watch and make it recognize gesture over the watch surface.

These interaction extension break through the limits of screen size though,  but its still a bimanual interaction, still tiredness and inconvenience for users.
%一些研究给出了不同的思路,他们把局限于智能手表上的交互扩展到了手表的外围\cite{Knibbe:2014:EIS:2559206.2581315,Kratz:2009:HEA:1613858.1613912}、手表的表带\cite{Perrault:2013:WSG:2470654.2466192}、甚至用户的皮肤上\cite{Ogata:2015:SSG:2735711.2735830}。这些扩展式的交互虽然打破了屏幕大小的限制,但是并没有考虑到即便是没有了输入式交互的范围限制,依然需要双手来执行这些操作,依然是一种较为疲态、不够便捷的交互形式。

%\cite{kim2007gesture}在手表的周围增加了多个无线传感器,通过对手表屏幕上方的不同动作进行识别从而增加对手表交互的方式。然而这种方法仍依赖两只手进行完成,并且完全没有考虑过这种方法在单手交互上的扩展性。

%在文\cite{Yang:2015:EST:2815585.2815724}中设计了一个增强交互的方式,通过对佩戴手表的手部的手势进行识别来执行不同类型的交互。虽然这种设计考虑了手势在智能手表交互上的应用,但是这种方式不但没有简化在小屏幕设备上的交互逻辑,反而进一步增加了用户对执行这种隐式交互的学习成本。

This extension of interaction isn't limited on smart watch, \cite{Chen:2014:DEJ:2556288.2556955} augmented the smart phone interaction by smart watch, and \cite{Yang:2014:MIS:2638728.2638848} give us an implication for use watch position in space, essentially, it a kind of Virtual Reality development. By this, we gains the multiplicity of watch state and even smart phone and watch works simultaneously.
%对于扩展式的交互还不仅局限于手表上,文\cite{Chen:2014:DEJ:2556288.2556955}则将手表作为一个手机的位于身体手部位置的扩展辅助传感器,文\cite{Yang:2014:MIS:2638728.2638848}考虑了智能手表的空间位置,本质上将智能手表扩展往一种穿戴式的体感交互平台发展\cite{fuquanjun2015},这些研究进一步增加了对手表状态的识别进而增强在智能手机上交互的多样性。

In summary, although the interactive mode research coverage broad in smart watches, but these existing research on smart watches re generally avoided and did not consider the following questions:
%综上所述,尽管对手表现有交互模式的研究覆盖面很广,但现有的这些关于智能手表的研究中普遍回避了、且没有考虑下述的问题:
\begin{itemize}
    \kaishu
    \item The theoretical interaction techniques could not compatible with existing smart watch interactive mode which is not universal.
    %与现有智能手表的交互模式共存的兼容性较低,无法通用;
    \item These techniques is not broad enough, most of them didn't make a full analyze on a series of products and the design itself always give up on affordance.
    %没有针对一个系列的产品进行全面可行的应用分析,给出的设计适用面很窄,通常一个方法只能完成一件事情,而其设计本身又不具备功能可见性;
    \item Most theoretical techniques are suitable for bimanual interaction, there is no obvious context for increasing user stickiness on smart watches;
    %给出的设计普遍只适用于双手交互,对增加用户在产品上的黏性没有明显帮助;
    \item Do not have much considering of the relationship between interaction mode and user interfaces, on the one hand due to the lower social acceptability, and the other hand is because of the interactive logic not too strong, learning costs of users too high.
    %在设计上没有考虑过交互模式与用户界面之间的关系,一方面是给出的交互方式的社会接受度不高,另一方面则是交互的逻辑性不强,不够直观,用户的学习成本高。
\end{itemize}

\cleardoublepage
